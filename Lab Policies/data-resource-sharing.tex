\section{Data and Resource Sharing
Plan}\label{data-and-resource-sharing-plan}

It is the philosophy of the group to make all data generated by the
laboratory or in collaboration with the laboratory available to the
widest potential audience with the least number of restrictions.

\subsection{Data Sharing}\label{data-sharing}

Publications arising from these studies where the PI is the
corresponding author will be published in journals meeting the gold open
access standards of free availability, and a permissive re-use license
(CC-BY or the like). All published data will be submitted to journals
indexed by the major biomedical indexing sites including PubMed and
PubMed Central. Typically pre-prints will be made available at or before
the time of submission through bioRxiv. At this point all data will be
shared publicly.

For all studies the raw data (as well as the relevant metadata) and the
computational algorithms used to generate statistical and graphical
summaries which were used to generate the final research data will be
made public. This data will be provided with a permissive (CC0 or CC-BY)
use license. All raw data, processed data and analysis code will be made
available on Github with the version at time of publication archived to
the web-accessible data repositories \href{http://zenodo.com}{Zenodo}
with a unique DOI. All next generation sequence files will be deposited
into the \href{https://www.ncbi.nlm.nih.gov/sra}{Short Read Archive} and
\href{http://http://www.ncbi.nlm.nih.gov/geo/\%7D}{Gene Expression
Omnibus} and the accession number will be quoted in the manuscript(s).
Finally data will be posted on (or linked to) the principal
investigator's website. If further data is obtained relevant to these
studies, which can be combined with thepublished data sets, these
evolving data sets will be added to the available online repositories.

\subsection{Resource Sharing}\label{resource-sharing}

All commercially purchased antibodies, reagents or kits will be clearly
described in publications, including catalog or stock numbers and RRID's
if permissible. DNA constructs will be deposited with Addgene and also
made available by the PI without restrictions upon request. Reagents,
fruit fly, cell or mouse transgenic lines generated during these studies
will be maintained in the principal investigator's laboratory and made
available freely and without restriction to any scientists, providing
sufficient amounts are available. The only exception to this is if a
reagent or line was provided to us with a restrictive materials transfer
agreement, in which case the recipient investigator will be directed to
the original source of the reagent.

\subsection{Notes}\label{notes}

\begin{itemize}
\tightlist
\item
  \textbf{Version 1.5.1}
\item
  Updated on January 29, 2019 by Dave Bridges
  \textless{}\href{mailto:dave.bridges@gmail.com}{\nolinkurl{dave.bridges@gmail.com}}\textgreater{}
\item
  The version numbering for this document is described in the
  \href{https://github.com/BridgesLab/Lab-Documents/blob/master/Lab\%20Policies/README.rst}{Lab
  Policies README}. See the
  \href{https://github.com/BridgesLab/Lab-Documents/blob/master/Lab\%20Policies/data-resource-sharing.md}{GitHub
  Repository} for more granular changes.
\end{itemize}
