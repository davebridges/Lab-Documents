\documentclass[]{article}
\usepackage{lmodern}
\usepackage{amssymb,amsmath}
\usepackage{ifxetex,ifluatex}
\usepackage{fixltx2e} % provides \textsubscript
\ifnum 0\ifxetex 1\fi\ifluatex 1\fi=0 % if pdftex
  \usepackage[T1]{fontenc}
  \usepackage[utf8]{inputenc}
\else % if luatex or xelatex
  \ifxetex
    \usepackage{mathspec}
  \else
    \usepackage{fontspec}
  \fi
  \defaultfontfeatures{Ligatures=TeX,Scale=MatchLowercase}
\fi
% use upquote if available, for straight quotes in verbatim environments
\IfFileExists{upquote.sty}{\usepackage{upquote}}{}
% use microtype if available
\IfFileExists{microtype.sty}{%
\usepackage{microtype}
\UseMicrotypeSet[protrusion]{basicmath} % disable protrusion for tt fonts
}{}
\usepackage[margin=1in]{geometry}
\usepackage{hyperref}
\hypersetup{unicode=true,
            pdftitle={General Statistics},
            pdfauthor={Dave Bridges},
            pdfborder={0 0 0},
            breaklinks=true}
\urlstyle{same}  % don't use monospace font for urls
\usepackage{color}
\usepackage{fancyvrb}
\newcommand{\VerbBar}{|}
\newcommand{\VERB}{\Verb[commandchars=\\\{\}]}
\DefineVerbatimEnvironment{Highlighting}{Verbatim}{commandchars=\\\{\}}
% Add ',fontsize=\small' for more characters per line
\usepackage{framed}
\definecolor{shadecolor}{RGB}{248,248,248}
\newenvironment{Shaded}{\begin{snugshade}}{\end{snugshade}}
\newcommand{\KeywordTok}[1]{\textcolor[rgb]{0.13,0.29,0.53}{\textbf{{#1}}}}
\newcommand{\DataTypeTok}[1]{\textcolor[rgb]{0.13,0.29,0.53}{{#1}}}
\newcommand{\DecValTok}[1]{\textcolor[rgb]{0.00,0.00,0.81}{{#1}}}
\newcommand{\BaseNTok}[1]{\textcolor[rgb]{0.00,0.00,0.81}{{#1}}}
\newcommand{\FloatTok}[1]{\textcolor[rgb]{0.00,0.00,0.81}{{#1}}}
\newcommand{\ConstantTok}[1]{\textcolor[rgb]{0.00,0.00,0.00}{{#1}}}
\newcommand{\CharTok}[1]{\textcolor[rgb]{0.31,0.60,0.02}{{#1}}}
\newcommand{\SpecialCharTok}[1]{\textcolor[rgb]{0.00,0.00,0.00}{{#1}}}
\newcommand{\StringTok}[1]{\textcolor[rgb]{0.31,0.60,0.02}{{#1}}}
\newcommand{\VerbatimStringTok}[1]{\textcolor[rgb]{0.31,0.60,0.02}{{#1}}}
\newcommand{\SpecialStringTok}[1]{\textcolor[rgb]{0.31,0.60,0.02}{{#1}}}
\newcommand{\ImportTok}[1]{{#1}}
\newcommand{\CommentTok}[1]{\textcolor[rgb]{0.56,0.35,0.01}{\textit{{#1}}}}
\newcommand{\DocumentationTok}[1]{\textcolor[rgb]{0.56,0.35,0.01}{\textbf{\textit{{#1}}}}}
\newcommand{\AnnotationTok}[1]{\textcolor[rgb]{0.56,0.35,0.01}{\textbf{\textit{{#1}}}}}
\newcommand{\CommentVarTok}[1]{\textcolor[rgb]{0.56,0.35,0.01}{\textbf{\textit{{#1}}}}}
\newcommand{\OtherTok}[1]{\textcolor[rgb]{0.56,0.35,0.01}{{#1}}}
\newcommand{\FunctionTok}[1]{\textcolor[rgb]{0.00,0.00,0.00}{{#1}}}
\newcommand{\VariableTok}[1]{\textcolor[rgb]{0.00,0.00,0.00}{{#1}}}
\newcommand{\ControlFlowTok}[1]{\textcolor[rgb]{0.13,0.29,0.53}{\textbf{{#1}}}}
\newcommand{\OperatorTok}[1]{\textcolor[rgb]{0.81,0.36,0.00}{\textbf{{#1}}}}
\newcommand{\BuiltInTok}[1]{{#1}}
\newcommand{\ExtensionTok}[1]{{#1}}
\newcommand{\PreprocessorTok}[1]{\textcolor[rgb]{0.56,0.35,0.01}{\textit{{#1}}}}
\newcommand{\AttributeTok}[1]{\textcolor[rgb]{0.77,0.63,0.00}{{#1}}}
\newcommand{\RegionMarkerTok}[1]{{#1}}
\newcommand{\InformationTok}[1]{\textcolor[rgb]{0.56,0.35,0.01}{\textbf{\textit{{#1}}}}}
\newcommand{\WarningTok}[1]{\textcolor[rgb]{0.56,0.35,0.01}{\textbf{\textit{{#1}}}}}
\newcommand{\AlertTok}[1]{\textcolor[rgb]{0.94,0.16,0.16}{{#1}}}
\newcommand{\ErrorTok}[1]{\textcolor[rgb]{0.64,0.00,0.00}{\textbf{{#1}}}}
\newcommand{\NormalTok}[1]{{#1}}
\usepackage{graphicx,grffile}
\makeatletter
\def\maxwidth{\ifdim\Gin@nat@width>\linewidth\linewidth\else\Gin@nat@width\fi}
\def\maxheight{\ifdim\Gin@nat@height>\textheight\textheight\else\Gin@nat@height\fi}
\makeatother
% Scale images if necessary, so that they will not overflow the page
% margins by default, and it is still possible to overwrite the defaults
% using explicit options in \includegraphics[width, height, ...]{}
\setkeys{Gin}{width=\maxwidth,height=\maxheight,keepaspectratio}
\IfFileExists{parskip.sty}{%
\usepackage{parskip}
}{% else
\setlength{\parindent}{0pt}
\setlength{\parskip}{6pt plus 2pt minus 1pt}
}
\setlength{\emergencystretch}{3em}  % prevent overfull lines
\providecommand{\tightlist}{%
  \setlength{\itemsep}{0pt}\setlength{\parskip}{0pt}}
\setcounter{secnumdepth}{5}
% Redefines (sub)paragraphs to behave more like sections
\ifx\paragraph\undefined\else
\let\oldparagraph\paragraph
\renewcommand{\paragraph}[1]{\oldparagraph{#1}\mbox{}}
\fi
\ifx\subparagraph\undefined\else
\let\oldsubparagraph\subparagraph
\renewcommand{\subparagraph}[1]{\oldsubparagraph{#1}\mbox{}}
\fi

%%% Use protect on footnotes to avoid problems with footnotes in titles
\let\rmarkdownfootnote\footnote%
\def\footnote{\protect\rmarkdownfootnote}

%%% Change title format to be more compact
\usepackage{titling}

% Create subtitle command for use in maketitle
\newcommand{\subtitle}[1]{
  \posttitle{
    \begin{center}\large#1\end{center}
    }
}

\setlength{\droptitle}{-2em}
  \title{General Statistics}
  \pretitle{\vspace{\droptitle}\centering\huge}
  \posttitle{\par}
  \author{Dave Bridges}
  \preauthor{\centering\large\emph}
  \postauthor{\par}
  \predate{\centering\large\emph}
  \postdate{\par}
  \date{May 9, 2014}


\begin{document}
\maketitle

{
\setcounter{tocdepth}{2}
\tableofcontents
}
\section{General Statistical Methods}\label{general-statistical-methods}

There are several important concepts that we will adhere to in our
group. These involve design considerations, execution considerations and
analysis concerns.

\subsection{Experimental Design}\label{experimental-design}

Where possible, prior to performing an experiment or study perform a
power analysis. This is mainly to determine the appropriate sample
sizes. To do this, you need to know a few of things:

\begin{itemize}
\tightlist
\item
  Either the sample size or the difference. The difference is provided
  in standard deviations. This means that you need to know the standard
  deviation of your measurement in question. It is a good idea to keep a
  log of these for your data, so that you can approximate what this is.
  If you hope to detect a correlation you will need to know the expected
  correlation coefficient.
\item
  The desired false positive rate (normally 0.05). This is the rate at
  which you find a difference where there is none. This is also known as
  the type I error rate.
\item
  The desired power (normally 0.8). This indicates that 80\% of the time
  you will detect the effect if there is one. This is also known as 1
  minus the false negative rate or 1 minus the Type II error rate.
\end{itemize}

We use the R package \textbf{pwr} to do a power analysis Champely
(2017). Here is an example:

\subsubsection{Pairwise Comparasons}\label{pairwise-comparasons}

\begin{verbatim}
## 
##      Two-sample t test power calculation 
## 
##               n = 3.07
##               d = 3
##       sig.level = 0.05
##           power = 0.8
##     alternative = two.sided
## 
## NOTE: n is number in *each* group
\end{verbatim}

This tells us that in order to see a difference of at least 3, with at
standard devation of 3.5 we need at least \textbf{3} observations in
each group.

\subsubsection{Correlations}\label{correlations}

The following is an example for detecting a correlation.

\begin{verbatim}
## 
##      approximate correlation power calculation (arctangh transformation) 
## 
##               n = 18.6
##               r = 0.6
##       sig.level = 0.05
##           power = 0.8
##     alternative = two.sided
\end{verbatim}

This tells us that in order to detect a correlation coefficient of at
least 0.6 (or an R\^{}2 of 0.36) you need more than \textbf{18}
observations.

\subsection{Corrections for Multiple
Observations}\label{corrections-for-multiple-observations}

The best illustration I have seen for the need for multiple observation
corrections is this cartoon from XKCD (see \url{http://xkcd.com/882/}):

\begin{figure}[htbp]
\centering
\includegraphics{http://imgs.xkcd.com/comics/significant.png}
\caption{Significance by XKCD. Image is from
\url{http://imgs.xkcd.com/comics/significant.png}}
\end{figure}

Any conceptually coherent set of observations must therefore be
corrected for multiple observations. In most cases, we will use the
method of Benjamini and Hochberg since our p-values are not entirely
independent. Some sample code for this is here:

\begin{verbatim}
##   unadjusted adjusted
## 1      0.023   0.0575
## 2      0.043   0.0700
## 3      0.056   0.0700
## 4      0.421   0.4210
## 5      0.012   0.0575
\end{verbatim}

\section{Session Information}\label{session-information}

\begin{Shaded}
\begin{Highlighting}[]
\KeywordTok{sessionInfo}\NormalTok{()}
\end{Highlighting}
\end{Shaded}

\begin{verbatim}
## R version 3.4.2 (2017-09-28)
## Platform: x86_64-apple-darwin15.6.0 (64-bit)
## Running under: macOS High Sierra 10.13.1
## 
## Matrix products: default
## BLAS: /Library/Frameworks/R.framework/Versions/3.4/Resources/lib/libRblas.0.dylib
## LAPACK: /Library/Frameworks/R.framework/Versions/3.4/Resources/lib/libRlapack.dylib
## 
## locale:
## [1] en_US.UTF-8/en_US.UTF-8/en_US.UTF-8/C/en_US.UTF-8/en_US.UTF-8
## 
## attached base packages:
## [1] stats     graphics  grDevices utils     datasets  methods   base     
## 
## other attached packages:
## [1] pwr_1.2-1           knitcitations_1.0.9 dplyr_0.7.4        
## [4] tidyr_0.7.2         knitr_1.17         
## 
## loaded via a namespace (and not attached):
##  [1] Rcpp_0.12.14       xml2_1.1.1         bindr_0.1         
##  [4] magrittr_1.5       R6_2.2.2           rlang_0.1.4       
##  [7] bibtex_0.4.2       plyr_1.8.4         httr_1.3.1        
## [10] stringr_1.2.0      tools_3.4.2        htmltools_0.3.6   
## [13] yaml_2.1.15        rprojroot_1.2      digest_0.6.12     
## [16] assertthat_0.2.0   tibble_1.3.4       bindrcpp_0.2      
## [19] purrr_0.2.4        RefManageR_0.14.20 glue_1.2.0        
## [22] evaluate_0.10.1    rmarkdown_1.8      stringi_1.1.6     
## [25] compiler_3.4.2     backports_1.1.1    lubridate_1.7.1   
## [28] jsonlite_1.5       pkgconfig_2.0.1
\end{verbatim}

\section{References}\label{references}

\protect\hyperlink{cite-pwr}{{[}1{]}} S. Champely. \emph{pwr: Basic
Functions for Power Analysis}. R package version 1.2-1. 2017. URL:
\url{https://CRAN.R-project.org/package=pwr}.


\end{document}
